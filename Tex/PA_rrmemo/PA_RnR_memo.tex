%\title{Overleaf Memo Template}
% Using the texMemo package by Rob Oakes
\documentclass[a4paper,11pt]{texMemo}
\usepackage[english]{babel}
\usepackage{graphicx, lipsum}

%% Edit the header section here. To include your
%% own logo, upload a file via the files menu.
\memosubject{Manuscript Revisions for \emph{Political Analysis}}
\memoto{PA-2020-075}
\memofrom{Generative Dynamics of Supreme Court Citations: Analysis with a New Statistical Network Model}
\memodate{\today}


\begin{document}
\maketitle

\noindent We are grateful for the opportunity to revise and resubmit our manuscript. In this memo
we have separated the reviewers' comments into separate criticisms and
suggestions. Under each comment, we describe how we have revised the manuscript in
response to the feedback provided. We generally agree with the criticisms offered, and think
that our manuscript has improved substantially as a result of incorporating this feedback. 



\section*{Reviewer: 1}

\noindent \textbf{R1.1:} \emph{ In terms of estimation, it reads as if the smaller networks used MCMLE and the later the simulated annealing approach. Is that right? If so, does the simulated approach arrive at the same values for the smaller year networks as the MCMLE? Showing that might help validate the approach for the bigger/later year networks. At the very least it might mitigate my next point on the overtime changes.   }\\

\noindent \textbf{Addressed:}   \\

\noindent  \textbf{R1.2:} \emph{ For at least a few of the variables --- reciprocity, gwidegree, and Justices in majority --- we see a substantial change in recent periods. Does this have anything to do with the change in estimation strategy? How convinced can we be that these fairly big changes are substantively driven? }\\

\noindent \textbf{Addressed:}   \\

\noindent \textbf{R1.3:} \emph{ Relatedly, why isn’t overtime change in the covariates discussed? The authors instead choose to make broad statements: “in/consistently significant” over the period of study.  This seems to be a wasted opportunity and contrary to their motivation (a la Shalizi and Rinaldo 2013). If the authors want to make a statement about whether the covariate generally holds, then why not run a single model across all the years? Otherwise, wouldn’t it make sense to discuss these over time changes, at least when they appear to be quite meaningful?  }\\

\noindent \textbf{Addressed:}   \\

\noindent \textbf{R1.4:} \emph{ Not to belabor the point, but the motivation for the exogenous and endogenous terms are nicely situated in the literature, but the discussion of the results are quite anticlimactic. It merely notes that some results are generally in line with expectations while others are not (the ideology finding being terribly interesting and surprising!). I’d like to see a more substantive engagement with the results, particularly any trends / disruptions in the series of coefficients. This has the potential to be a much bigger contribution to the judicial literature than the authors acknowledge.  }\\

\noindent \textbf{Addressed:}   \\


\noindent \textbf{R1.5:} \emph{ While this is a well-executed paper (and visually beautiful), it’s not clear where the contribution is meant to be. Is this paper an introduction of a new model or an application of the ERGM to citation data?  If this is a new network model that would merit its own name, c-ERGM, what other networks might this model apply to?  Does it really make sense to have a model that just applies to this one judicial network...? I suppose this model works for any citations --- e.g., academic papers -- so maybe discussing a bit of that literature and data would help. In short, the change to the ERGM is slight, even when combining that with the necessary simulated estimation approach, so telling us why these changes are so useful beyond this network could drive home the methodological contribution.  }\\

\noindent \textbf{Addressed:}   \\

\noindent \textbf{R1.6:} \emph{ Alternatively, one could think about this paper more in terms of its substantive contributions. The front end is deeply ingrained in the citations and judicial literatures, as are the data, of course, but little attention is given in the end to what we learn from this paper. It seems like the authors decided not to engage either frame fully, which leaves the contribution lacking on both. Having said that, I think one could go either way with this paper, and I liked a lot about it on both fronts. }\\

\noindent \textbf{Addressed:}   \\

\section*{Reviewer: 2}


\noindent \textbf{R2.1:} \emph{ }\\

\noindent \textbf{Addressed:}  \\


\end{document} 
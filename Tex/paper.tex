\documentclass[headsepline=true, abstracton]{scrartcl}
%\usepackage[ngerman]{babel}
\usepackage[utf8]{inputenc}
\usepackage[T1]{fontenc}
\usepackage{amssymb}
\usepackage{amsmath}
\usepackage{amsthm}
\usepackage{enumerate}
\usepackage{verbatim}
\usepackage[a4paper,text={140mm,215mm},centering,headsep=10mm,footskip=10mm]{geometry}
\usepackage[urlcolor=black,colorlinks=true,linkcolor=black,citecolor=black,bookmarks]{hyperref}
\usepackage{aliascnt}
\usepackage{lmodern}
\usepackage{mdwlist}
\usepackage{natbib}
\usepackage{listings}
\usepackage[table,xcdraw]{xcolor}
\usepackage[activate]{pdfcprot}
\usepackage{graphicx}
\usepackage{slashed}
\usepackage{mathrsfs}
\usepackage{geometry}
\usepackage{float}
\usepackage{oldgerm}
\usepackage{setspace}
\usepackage{dsfont}
%\usepackage{mathtools}
\usepackage[all]{xy}
\usepackage{cite}
%\usepackage{amsmath}
\usepackage{amsfonts}
\usepackage{amssymb}
\usepackage{algorithm2e}
\usepackage{url}
\pagestyle{headings}
\newcommand{\myclearpage}{\clearpage}
\newcommand\independent{\protect\mathpalette{\protect\independenT}{\perp}}
\def\independenT#1#2{\mathrel{\rlap{$#1#2$}\mkern2mu{#1#2}}}
\newenvironment{gelaber}{}{}
\newenvironment{preamble}{}{}
\newcommand{\tostar}{\overset{*}{\lower0.5em\hbox{$\smash{\scriptscriptstyle\rightharpoonup}$}}}
\newtheorem{mydef}{Definition}
\newtheorem{bem}{Bemerkung}
\usepackage{tikz}
\newcommand\circlearound[1]{%
  \tikz[baseline]\node[draw,shape=circle,anchor=base] {#1} ;}
 
  
\begin{document}



\renewcommand{\refname}{Bibliography}


\onehalfspacing
\setlength{\headsep}{15mm}


\thispagestyle{plain}

\title{\Large Analyzing the Supreme Court Citation Network}
\maketitle

\begin{abstract}
\noindent 
\end{abstract}


 \section{Introduction}
 
 
 
 
Majority opinions written by the United States Supreme Court exercise their authority and influence, in part, through their roles as precedents in future Supreme Court decisions and opinions. The findings regarding the extent and exact nature of the influences of precedent have been mixed, but the balance of the literature finds that past decisions exert some form of influence on the justices' decision making \citep{knight1996norm,gillman2001s,richards2002jurisprudential,hansford2006politics,bailey2008does,bailey2011constrained}. Despite a considerable body of research that focuses on the way in which relevant precedents shape decision making on the Court, relatively little work has focused on understanding which past opinions are cited by an opinion. Our focus in this paper is to provide what is, to our knowledge, the first comprehensive analysis of exactly which cases are cited by a case. We follow an emerging body of work on legal citations, and treat the system of citations as a network \citep[e.g., ][]{fowler2007network, fowler2008authority,bommarito2009law,lupu2012precedent,pelc2014politics}. 

We are not the first to study what predicts citations on the Supreme Court...


We are the first to examine at the level of the citation, and, as a result, be able to incorporate complex dependence into our explanation of citations.


\section{Prior work on court citations}

Measures of citations have been used in several recent studies of courts...\citet{hinkle2016transmission} study the citations between US state supreme courts.

\begin{itemize}

\item \citet{clark2010locating} Estimates a latent coordinate model of supreme court opinions based on the network of supreme court citations. Find that the majority opinion falls at the ideal point of the median member of the majority coalition. Account for negative citations. WE COULD USE THIS DATASET AS A SECOND APPLICATION

\item \citet{cross2010determinants} Look at the total number of citations to Supreme Court opinions over a ten year period of the Rehnquist Court. (case level).  Found case issue area, opinion ideology, opinion author, vote margin was not significant. Negative citations decreased future positive citations.

\item \citet{black2013citation} Study the number of citations to a case over time, in order to understand the lifecycle of citation rates.  Show that the rate of citations to a case goes down very quickly. THE EFFECT OF AN INDEPENDENT VARIABLE ON A PRECEDENT IS DEPENDENT UPON THE AGE OF THE CASE.

\item \citet{benjamin2012standing} Study the propensity for cases to be overruled and cited in other negative ways. They find that cases with large and broad majorities rare less likely to be cited negatively.

\item \citet{spriggs2001explaining} Find that cases are more likely to be overruled when the Court is ideologically distant from the median justice in the majority coalition that decided the case, when the legal issues under consideration are complex, whn the cases ha been negatively treated previously, when there were many concurring opinions, and less likely to be overruled if it was a unanimous coalition.

\item \citet{lupu2013strategic} The hub score, which is a measure of the degree to which a case cites other important cases increases with salience [GET THE AMICUS BRIEFS, NYT COUNTS]. Lower if stautory cases. Increases with ideological variance of the majority coalition.  Increases with the number of concurring opinions. 

\end{itemize}


\section{Network Approaches to Studying Citations}

\textbf{TED}, would you write a section that covers the ways in which network scholars have studied citations in the past, focusing on scientific papers, patents, and court cases? 


\section{Complex Dependence in Court Citations}

\textbf{Transitivity:} Discussing past edges in the citation network leads to transitivity. If i cites j, it is also likely to discuss the cases on which j is based.

The statute thus addresses the risk of a morally unjustifiable death sentence, not by minimizing it as precedent unmistakably requires, but by guaranteeing that in equipoise cases the risk will be realized, by ?placing a ?thumb [on] death?s side of the scale,? ? Sochor v. Florida, 504 U. S. 527, 532 (1992) (quoting Stringer v. Black, 503 U. S. 222, 232 (1992); alteration in original).

https://supreme.justia.com/cases/federal/us/548/163/dissent2.html


``Never before the decision in Union Gas had we suggested that the bounds of Article III could be expanded by Congress operating pursuant to any constitutional provision other than the Fourteenth Amendment. Indeed, it had seemed fundamental that Congress could not expand the jurisdiction of the federal courts beyond the bounds of Article III. Marbury v. Madison, 1 Cranch 137 (1803). The plurality's citation of prior decisions for support was based upon what we believe to be a misreading of precedent. See Union Gas, 491 U. S., at 40-41 (SCALIA, J., dissenting). The plurality claimed support for its decision from a case holding the unremarkable, and completely unrelated, proposition that the States may waive their sovereign immunity, see id., at 14-15 (citing Parden v. Terminal Railway of Ala. Docks Dept., 377 U. S. 184 (1964)), and cited as precedent propositions that had been merely assumed for the sake of argument in earlier cases, see 491 U. S., at 15 (citing Welch v. Texas Dept. of Highways and Public Transp., 483 U. S., at 475-476, and n. 5, and County of Oneida v. Oneida Indian Nation of N. Y., 470 U. S., at 252).''

https://supreme.justia.com/cases/federal/us/517/44/case.html


\textbf{Mutuality:} When deciding a set of related cases, the courts opinions regularly reference each other in order to reinforce their arguments...

On a more specific level, Western argues that flight engineers must meet the same stringent qualifications as pilots, and that it was therefore quite logical to extend to flight engineers the FAA's age 60 retirement rule for pilots. Although the FAA's rule for pilots, adopted for safety reasons, is relevant evidence in the airline's BFOQ defense, it is not to be accorded conclusive weight. Johnson v. Mayor and City Council of Baltimore, ante at 472 U. S. 370-371. The extent to which the rule is probative varies with the weight of the evidence supporting its safety rationale and "the congruity between the . . . occupations at issue." Ante at 472 U. S. 371. In this case, the evidence clearly established that the FAA, Western, and other airlines all recognized that the qualifications for a flight engineer were less rigorous than those required for a pilot. [Footnote 28]

https://supreme.justia.com/cases/federal/us/472/400/case.html

The city, supported by several amici, argues for affirmance nonetheless. It asserts first that the federal civil service statute is not just a federal retirement provision unrelated to the ADEA, but in fact establishes age as a BFOQ for federal firefighters based on factors that properly go into that determination under the ADEA, see Western Air Lines, Inc. v. Criswell, post p. 472 U. S. 400. Second, the city asserts, a congressional finding that age is a BFOQ for a certain occupation is dispositive of that determination with respect to nonfederal employees in that occupation. We consider each of these contentions in turn.

\url{https://supreme.justia.com/cases/federal/us/472/353/case.html#370}

\textbf{Also popularity} If j served as precedent for i, j is authoritative regarding the rule of i, and should also be cited. Citations beget citations. Citation confers authority. Cases are even discussed in opinions in terms of the extent of influence they have had on other cases,

"The first case in which this Court struck down a statute under the Equal Protection Clause of the Fourteenth Amendment was Strauder v. West Virginia, 100 U. S. 303, decided in the 1879 Term. [Footnote 2/1] In the 1961 Term, we squarely held that the manner of apportionment of members of a state legislature raised a justiciable question under the Equal Protection Clause, Baker v. Carr, 369 U. S. 186. That case was followed by numerous others, e.g.: that one person could not be given twice or 10 times the voting power of another person in a state-wide election merely because he lived in a rural area..."

https://supreme.justia.com/cases/federal/us/400/112/case.html

\textbf{Sociality} Citations sent beget citations sent.  For each case discussed, often related cases to discuss, even those that do not cite each other. For each case that applies, there is likely a case that needs to be discussed in terms of why it does not apply. Justices often clarify not only those legal principles that apply, but often those that do not. When multiple lines of legal reasoning are drawn upon to justify a decision, often more are discussed in terms of their lack of applicability. 

"1. The "overbreadth" doctrine is not applicable here. There is nothing in the record to indicate that \S 28.04 will have any different impact on any third parties' interests in free speech than it has on appellees' interests, and appellees have failed to identify any significant difference between their claim that \S 28.04 is invalid on overbreadth grounds and their claim that it is unconstitutional when applied to their signs during a political campaign. Thus, it is inappropriate to entertain an overbreadth challenge to \S 28.04. Pp. 466 U. S. 796-803."

https://supreme.justia.com/cases/federal/us/466/789/

''No allegations of racial or class-based invidiously discriminatory animus are required to establish a cause of action under the first part of  1985(2). The statutory provisions now codified at \S 1985 were originally enacted as \S 2 of the Civil Rights Act of 1871, and the substantive meaning of the 1871 Act has not been changed. The provisions relating to institutions and processes of the Federal Government (including the first part of \S 1985(2)) -- unlike those encompassing activity that is usually of primary state concern (including the second part of \S 1985(2) and the part of \P 1985(3) involved in Griffin, supra -- contain no language requiring that the conspirators act with intent to deprive their victims of the equal protection of the laws. Thus, the reasoning of Griffin is not applicable here, and, given the structure of \S 2 of the 1871 Act, it is clear''

https://supreme.justia.com/cases/federal/us/460/719/


 
 \section{The Exponential Random Configuration Model}
Let $c(t)\in \{0,1\}^N$ be a vector indicating which Supreme Court case has been cited at time $t$, where $c_i(t)=1, i \in \{1, \dots , N\}$ indicates that the $i$th case has been cited at time $t$ and $c_i(t)=0$ indicates that the $i$th case has not been cited at time $t$. Furthermore, let
$$\mathcal{C}_t(N)=\{c(t)\in \{0,1\}^N: c_i(t)\in \{0,1\} \}$$ 
be the set of all possible citation combinations at time $t$. Note that the cardinality of $\mathcal{C}_t(N)$ increases exponentially for every newly added case, which results in $2^N$ elements.\\
The probability function of the ERCM is defined as
\begin{equation}
P_{\theta}(c(t)~|~c(t-))=\cfrac{\exp\big(\theta^T\cdot h\big(c(t)~|~c(t-)\big)\big)}{\sum_{c(t)^*\in \mathcal{C}}\exp\big(\theta^T\cdot h\big(c(t)~|~c(t-)\big)\big)}
\label{ercm}
\end{equation}
where $c(t-)\in \{0,1\}^{N \times (t-1)}$ is a matrix that indicates which cases have been citing in each other before time $t$, $\theta \in \mathbb{R}^q$ is a q-dimensional vector of parameters,  $h: \mathcal{C}_t(N) \to \mathbb{R}^q ~, ~ \c(t) \to (h_1(c(t)), \dots , h_q(c(t)))^T$ is a q-dimensional vector of different statistics and $\kappa(\theta) := \sum_{c(t)^*\in \mathcal{C}}\exp(\theta^T\cdot h(c(t)|c(t-)))$ is a normalization constant that ensures that (\ref{ercm}) defines a probability function on $\mathcal{C}_t$.\\[0.3cm]

% Network Statistics
The generative process of a model are informed by
the decision regarding which network statistics $h(\cdot)$ are incorporated. We include the following statistics for the Supreme Court citation network:
$$h_{edges}:\mathcal{C}(N)\to \mathbb{R}~~~, ~~~c(t) \to \sum_{i=1}^Nc_i(t)$$
the number of citations made at time $t$. 
$$h_{outstar}:\mathcal{C}(N)\to \mathbb{R}~~~, ~~~c(t) \to \sum_{j<i}^Nc_i(t)\cdot c_j(t) \cdot \sqrt{\dfrac{(t-a)(t-a-b)}{t^2}}$$
the number of weighted outstars occuring at time $t$. We argue that it should be more likely to cite more recent cases than cases that have been decided further in the past. For the weight 
$$w(a,b):= \sqrt{\dfrac{(t-a)(t-a-b)}{t^2}}$$
we define $a$ and $b$ as the elapsed time since case $i$ and $j$ have been introduced to the network.
$$h_{triangle}:\mathcal{C}(N)\to \mathbb{R}~~~, ~~~c(t) \to \sum_{j<i}^Nc_i(t)\cdot c_j(t) \cdot
c_j(t_{-i}) \cdot w(a,b)$$
where $c_j(t_{-i})$ indicates whether case $j$ was cited at the time case $i$ was introduced into the network. Just as for the outstar statistic, we include a weighting factor to favor more recent cases. \\[0.3cm]


%Change Statistic
The individual entries $c_i(t)$ can be taken as a manifestation of single Bernoulli variables $C_i(t)$. This interpretation allows the following calculation regarding the conditional distribution of $C_i(t)$:
%
\begin{eqnarray*}
\dfrac{P_{\theta}(C_i(t)=1 ~|~ C_i(t)^c=c_i(t)^c)}{P_{\theta}(C_i(t)=0 ~|~ C_i(t)^c=c_i(t)^c)} &=&
\dfrac{P_{\theta}(C_i(t)=1 ~,~ C_i(t)^c=c_i(t)^c)}{P_{\theta}(C_i(t)=0 ~,~ C_i(t)^c=c_i(t)^c)} \\
                           &=&\dfrac{P_{\theta}(C(t)= c_i^+(t))}{P_{\theta}(C(t)=c_i^-(t))}\\
                           &=&\dfrac{\exp(\theta^T \cdot h(c_i^+(t)~|~c(t-)))}{\exp(\theta^T \cdot h(c_i^-(t)~|~c(t-)))}\\
                           &=& \exp(\theta^T \cdot (h(c_i^+(t)~|~c(t-)) - h(c_i^-(t)~|~c(t-)))
\end{eqnarray*}
%
This implies the following equation:
%
\begin{equation}
\text{logit}(P_{\theta}(C_i(t)=1 ~|~ C_i(t)^c=c_i(t)^c))= \theta^T \cdot (h(c_i^+(t)~~|c(t-)) - h(c_i^-(t)~|~c(t-)))
\label{Logit}
\end{equation}
In the equation above the following notations were used:
%
\begin{itemize}
\item $c_i^+(t)$ emerges from $c(t)$, while assuming $c_i(t)=1$
\item $c_i^-(t)$ emerges from $c(t)$, while assuming $c_i(t)=0$
\item The condition $C_i(t)^c=c_i(t)^c$ is short for: $C_j(t)=c_j(t)$ for all $j\in \{1,\dots,N\}$ with $i \neq j$
\item The expression $(\Delta c_i)(t):=h(c_i^+(t)~|~c(t-)) - h(c_i^-(t)~|~c(t-))$ is called the \textit{change statistic}. The $k$th component of $(\Delta c_i)(t)$ captures the difference between citation networks $c_i^+(t)$ and $c_i^-(t)$ on the $k$th integrated statistic in the model
\end{itemize}


\section{Estimation}

% Pseudo Likelihood
\subsection*{Maximum Pseudo-Likelihood Estimator}
One can assume that the dyads are independent of each other, which means that
the random variables $C_i(t)$ inside the random vector $C(t)$ are independent of each other.
In this case, the equation (\ref{Logit}) reduces to
$$logit(P_{\theta}(C_i(t) = 1)) = \theta^T \cdot (\Delta c_i)(t)$$
This corresponds with the logistic regression approach, where the observations of
the dependent variables are simply edge values of the observed citation vector,
and the observations of the covariate values are given as the scores of every single
change statistic. Therefore, the resulting likelihood function is of the following form:
\begin{equation}
\text{lik}(\theta)= P_{\theta}(C(t)=c(t))= \prod_{i} \dfrac{ \exp \left(\theta^T \Delta(c_i)(t) \right)}{1+\exp \left(\theta^T \Delta(c_i)(t) \right)}
\label{PseudoLik}
\end{equation}

\subsection*{Maximum Likelihood Estimator}
The more rigorous technique is to estimate the parameters directly with the log-likelihood function derived from (\ref{ercm}), which has the following form:
%
\begin{equation}
\text{loglik}(\theta)=\theta^T \cdot h(c(t)| c(t-))-\log(\kappa(\theta))
\label{loglik}
\end{equation}
%
The problem resulting from estimating the parameters with (\ref{loglik}) is that the term
%
$$\kappa(\theta):= \sum_{c(t)^* \in \mathcal{C}(N)} \exp(\theta^T \cdot h(c(t)^*|c(t-)))$$ 
%
which sums up the weighted statistics of all possible binar vectors of length $N$, has to be evaluated. However, the cardinality of $\mathcal{C}(N)$ (\#$(\mathcal{C})=2^N$) is incredibly large and a direkt calculation of this sum is for already small $N$ not feasible. \\[0.3cm]
An solution for this limitation is based on the following consideration: Fix a vector of parameters $\theta_0 \in \Theta$ from the underlying parameter range $\Theta$ and compute for $\theta \in \Theta$ the expected value
%
\begin{eqnarray*}
\mathbb{E}_{\theta_0}\left[ \exp\left((\theta - \theta_0)^T \cdot \Gamma(C(t))\right) \right]&=& 
\sum_{c(t) \in \mathcal{C}(N)}\exp\left((\theta - \theta_0)^T \cdot \Gamma(c(t))\right)\cdot \mathbb{P}_{\theta_0}(C(t)=c(t))\\
&=& \sum_{c(t) \in \mathcal{C}(N)}\exp\left((\theta - \theta_0)^T \cdot \Gamma(c(t))\right)\cdot 
\frac{\exp(\theta_0^T \cdot \Gamma(c(t)))}{\kappa(\theta_0)}\\
&=& \frac{1}{\kappa(\theta_0)} \sum_{c(t) \in \mathcal{C}(N)}\exp\left(\theta^T \cdot \Gamma(c(t))\right)\\
&=&\frac{\kappa(\theta)}{\kappa(\theta_0)}
\end{eqnarray*}
%
This equation offers the following possibility: If one draws $L$ random vectors $c^{(1)}(t), \dots ,c^{(L)}(t)$ out of a distribution $\mathbb{P}_{\theta_0}$ appropriately, one gets with the law of big numbers and a big enough sample $L$ the following relation:
%
\begin{equation}
\frac{1}{L}\cdot \sum_{i=1}^{L}  \exp\left((\theta - \theta_0)^T \cdot \Gamma(c^{(i)}(t))\right)
~~\approx~~ \mathbb{E}_{\theta_0}\left[ \exp\left((\theta - \theta_0)^T \cdot \Gamma(C(t))\right) \right] = \frac{\kappa(\theta)}{\kappa(\theta_0)}
\label{konver}
\end{equation}
%
This approximate can then be used to approximate the log likelihood function.\\[0.4cm]
Next, we will discuss how a sufficient number of suitable drawings $c^{(1)}(t), \dots ,c^{(L)}(t)$ can be sampled from the distribution $\mathbb{P}_{\theta_0}$. \\
For this purpose, the Markov Chain Monte Carlo (MCMC) methods can be used.
\subsection*{Gibbs sampling for the ERCM}
To be able to compute the approximate likelihood function one needs a sufficiently large number of random vectors from the distribution $\mathbb{P}_{\theta_0}$. Snijders \citet{Snijders.2002b} introduces an approach to sample random networks for the ERGM framework by using \textit{MCMC methods}. We adapt this approach for sampling appropriate binary vectors for the ERCM. \\[0.4cm]
\textit{Gibbs sampling}\\
%Choose any vector $c^{(0)}(t) \in \mathcal{C}(N)$ (e.g. observed vector). Afterwards, the length $L$ of the respective sub-sequence is determined. For $k \in \{0,...,L-1\}$ execute the following steps recursively (here the vector in its $k$th iteration is denoted as $c^{(k)}(t)$):\\

\begin{algorithm}[H]
 Choose any vector $c^{(0)}(t) \in \mathcal{C}(N)$ (e.g. observed vector)\\
 \For{i in 1:N}{
  Compute $\pi:= \cfrac{\exp(\theta^T\cdot \Delta(c_i)(t))}{1+\exp(\theta^T\cdot \Delta(c_i)(t))}$\\
  Draw a random number Z from Bin(1,$\pi$)\\
  \eIf{Z=1}{
   set $c^{(k+1)}_i = 1$ and $c^{(k+1)}_j=c^{(k)}_j$, if $i\neq j$
   }{
   set $c^{(k+1)}_i = 0$ and $c^{(k+1)}_j=c^{(k)}_j$, if $i\neq j$
  }
 }
 Start all over using $c^{(k+1)}$\\[0.3cm]
 \caption{Simulation of vectors of $\mathbb{P}_\theta$ using Gibbs sampling}
\end{algorithm}
\vspace{0.5cm}
%\begin{enumerate}
%\item Randomly choose a number $i \in \{1,\dots, N\}$
%\item Compute using the likelihood the value 
%$$\pi:= \mathbb{P}_{\theta}(C_{i}(t)=1 | C_{i}^c(t)=(c_{i}^{(k)}(t))^c)=\cfrac{\exp(\theta^T\cdot \Delta(c_i)(t))}{1+\exp(\theta^T\cdot \Delta(c_i)(t))}$$
%\item Draw a random number $Z$ from Bin$(1, \pi)$. %If
%\begin{itemize}
%\item $Z=0$, define $c^{(k+1)}(t)$ via
%$$c_{p}^{(k+1)}(t)=\begin{cases}
%0& \text{if}~ p=i \\
%c_{p}^{(k)}(t) &\text{if}~ p \neq i 
%\end{cases}$$
%\item $Z=1$, define $c^{(k+1)}(t)$ via
%$$c_p^{(k+1)}(t)=\begin{cases}
%1& \text{if}~ p=i \\
%c_p^{(k)}(t) &\text{if}~ p \neq i 
%\end{cases}$$
%\end{itemize}
%\item Start at step 1 with $c^{(k+1)}(t)$.
%\end{enumerate}
\noindent Using the depicted algorithm, a sequence of random vectors $c^{(0)}(t),...,c^{(L)}(t)$ can be simulated. Since the orignial vector was chosen randomly and the first simulated vectors are very dependent on the chosen mvector (only one entry is changed per iteration!), usually the first $B$ vectors, where $N \ll B \ll L$, are discarded as the so called \textit{Burn-In}.
\subsection*{Metropolis Hastings for the ERCM}\label{networksimulation}
Choose any vector $c^{(0)}(t) \in \mathcal{C}(N)$ to start with (e.g., the observed vector). For $k \in \{0,...,L-1\}$ recursively proceed as follows:\\
\begin{enumerate}
\item Randomly choose a number $i\in \{1,\dots, N\}$
\item Compute, using the equation (\ref{Logit}) the value
$$\pi := \dfrac{\mathbb{P}_{\theta}(C_{i}(t) \neq c_{i}^{(k)}(t) ~| ~C_i(t)^c=c_i(t)^c)}{\mathbb{P}_{\theta}(C_{i}(t) = c_{i}^{(k)}(t) ~| ~C_i(t)^c=c_i(t)^c)}$$
\item Define $\delta:= \min\{1, \pi\}$ and draw a random number $Z$ from Bin$(1, \delta)$. If
\begin{itemize}
\item $Z=0$, let $c^{(k+1)}(t) := c^{(k)}(t)$ 
\item $Z=1$, define $c^{(k+1)}(t)$ as
$$c_{p}^{(k+1)}(t)=\begin{cases}
1-c_{p}^{(k)}(t)& \text{if}~ p=i \\
c_{p}^{(k)}(t) &\text{if}~ p \neq i 
\end{cases}$$
\end{itemize}
\item Start at step 1 with $c^{(k+1)}(t)$.
\end{enumerate}
The first $B \ll L$ vectors are discarded as Burn-In.


 \section{Results}
  \subsection*{Descriptive Results}
The supreme court citation network from $1937 - 2005$ consists of $8817$ cases which got voted at $2116$ different time points. The network has a total of $93,263$ ties, of which 452 are mutual. The number of triangles in the network is $211,855$. The in- and outdegree distribution is visualized in figures \ref{indegree_dist} and \ref{outdegree_dist}. The maximum indegree is $190$ and the maximum outdegree is $159$.

\begin{table}[H]
\centering
\begin{tabular}{|
>{\columncolor[HTML]{C0C0C0}}l |l|l|l|}
\hline
{\color[HTML]{333333} } & \cellcolor[HTML]{C0C0C0}{\color[HTML]{333333} Terms} & \cellcolor[HTML]{C0C0C0}{\color[HTML]{333333} Total Number Cases} & \cellcolor[HTML]{C0C0C0}{\color[HTML]{333333} Cases/Term} \\ \hline
CE Hughes*              & 1937 - 1941                                          & 629                                                               & 125.5                                                     \\ \hline
HF Stone                & 1942 - 1946                                          & 766                                                               & 153.2                                                     \\ \hline
FM Vinson               & 1946 - 1953                                          & 788                                                               & 98.5                                                      \\ \hline
E Warren                & 1954 - 1969                                          & 2159                                                              & 127.0                                                     \\ \hline
WE Burger               & 1970 - 1986                                          & 2805                                                              & 155.8                                                     \\ \hline
W Rehnquist**            & 1987 - 2001                                          & 1670                                                              & 83.5                                                      \\ \hline
\end{tabular}
\caption[caption]{For the time range of interest (1937 - 2001) this table displays the chief justices, the time range they served as chief justice, the number of cases in their time range as well as the average number of cases per year.\\\hspace{\textwidth} * CE Hughes served as chief justice from 1930 - 1941. \\\hspace{\textwidth} ** W Rehnquist served as chief justice from 1987 - 2005.}
\label{my-label}
\end{table}

\begin{figure}[H]
\includegraphics[scale=0.5]{degree_distribution}
\caption{The in- and outdegree distribution of the Supreme Court Citation Network from 1937 - 2001. There are cases with an indegree >50, but they are not captured in this figure.}
 \label{degree_dist}
\vspace{-.25cm}
\end{figure}

\begin{figure}[H]
\includegraphics[scale=0.5]{number_cases}
\caption{Number of cases in each term. Different colors indicate different chief justices.}
 \label{number_cases}
\vspace{-.25cm}
\end{figure}  
  
 \begin{figure}[H]
\includegraphics[scale=0.5]{number_citations}
\caption{Number of citations for the 1937-2001 time period. Citations for cases prior 1937 are not considered in this figure. Different colors indicate different chief justices.}
 \label{number_citations}
\vspace{-.25cm}
\end{figure}  
  
  \begin{figure}[H]
\includegraphics[scale=0.5]{number_votes_supporting}
\caption{Number of Votes that were supporting cases between 1937-2001. Different colors indicate terms with different chief justices. }
 \label{number_supporting}
\vspace{-.25cm}
\end{figure}  
  
\newpage 



  
\subsection{Inferential Results}


\textbf{Model updates}
\begin{itemize}
\item Add sender issue area and receiver issue area node covariates (i.e., nodeifactor and nodeofactor applied to issue area). 
\item In the GLM, multiply each variable by the sender time covariate.
\item Write the bootstrap coefficients to a file, send to Bruce so he can write up a code for summarizing the over-time trends in effects.
\item Add a receiver variable (i.e., nodeicov) that equals the absolute difference between the maximum MQ score of a justice in the majority and a minimum MQ score of a justice in the majority.
\item Add a receiver variable (nodeicov) equivalent to the number of justices in the majority
 
\end{itemize}


\begin{table}[H]
\footnotesize
\centering
\begin{tabular}{|
>{\columncolor[HTML]{EFEFEF}}l |l|l|l|l|} 
\hline
                                                   & \cellcolor[HTML]{EFEFEF}Estimate & \cellcolor[HTML]{EFEFEF}Lower Bound & \cellcolor[HTML]{EFEFEF}Upper Bound & \cellcolor[HTML]{EFEFEF}Significance \\
                                                    \hline
Edges                                              & -5.533                           & -5.733                              & -5.388                              & *                                    \\ \hline
Instar(2)                                          & 0.031                            & 0.027                               & 0.036                               & *                                    \\ \hline
Outstar(2)                                         & 0.022                            & 0.020                               & 0.032                               & *                                    \\ \hline
Mutual                                             & 3.316                            & 2.622                               & 3.983                               & *                                    \\ \hline
Triangle                                           & 1.490                            & 1.410                               & 1.560                               & *                                    \\ \hline
Martin Quinn Score                                 & 0.080                            & 0.022                               & 0.126                               & *                                    \\ \hline
Same Issue Area Homophily                          & 1.378                            & 1.313                               & 1.451                               & *                                    \\ \hline
Year Difference                                    & -0.077                           & -0.090                              & -0.064                              & *                                    \\ \hline
$(\text{Year Differnce})^2$                        & 0.0029                           & 0.0025                              & 0.0032                              & *                                    \\ \hline
Receiver Abs Diff MQ Score in Majority             & -0.030                           & -0.048                              & -0.014                              & *                                    \\ \hline
Receiver Number Justices in Majority               & -0.115                           & -0.135                              & -0.089                              & *                                    \\ \hline
Receiver  Sender Year                              & 0.0007                           & 0.0006                              & 0.0009                              & *                                    \\ \hline
Sender Same Issue Area 2                           & 0.162                            & 0.105                               & 0.217                               & *                                    \\ \hline
Sender Same Issue Area 3                           & -0.317                           & -0.403                              & -0.236                              & *                                    \\ \hline
Sender Same Issue Area 4                           & 0.497                            & 0.426                               & 0.570                               & *                                    \\ \hline
Sender Same Issue Area 5                           & 0.332                            & 0.174                               & 0.495                               & *                                    \\ \hline
Sender Same Issue Area 6                           & 0.527                            & 0.402                               & 0.635                               & *                                    \\ \hline
Sender Same Issue Area 7                           & 0.327                            & 0.251                               & 0.394                               & *                                    \\ \hline
Sender Same Issue Area 8                           & 0.076                            & 0.021                               & 0.129                               & *                                    \\ \hline
Sender Same Issue Area 9                           & 0.261                            & 0.209                               & 0.315                               & *                                    \\ \hline
Sender Same Issue Area 10                          & 0.369                            & 0.301                               & 0.432                               & *                                    \\ \hline
Sender Same Issue Area 11                          & -0.046                           & -0.233                              & 0.151                               &                                      \\ \hline
Sender Same Issue Area 12                          & 0.0073                           & -0.075                              & 0.089                               &                                      \\ \hline
Sender Same Issue Area 13                          & 0.381                            & 0.162                               & 0.564                               & *                                    \\ \hline
Sender Same Issue Area 14                          & 0.151                            & -0.019                              & 0.300                               &                                      \\ \hline
Receiver Same Issue Area 2                         & 0.231                            & 0.166                               & 0.300                               & *                                    \\ \hline
Receiver Same Issue Area 3                         & -0.101                           & -0.209                              & -0.006                              & *                                    \\ \hline
Receiver Same Issue Area 4                         & 0.516                            & 0.425                               & 0.612                               & *                                    \\ \hline
Receiver Same Issue Area 5                         & 0.351                            & 0.183                               & 0.534                               & *                                    \\ \hline
Receiver Same Issue Area 6                         & 0.461                            & 0.284                               & 0.628                               & *                                    \\ \hline
Receiver Same Issue Area 7                         & 0.361                            & 0.281                               & 0.437                               & *                                    \\ \hline
Receiver Same Issue Area 8                         & 0.153                            & 0.097                               & 0.221                               & *                                    \\ \hline
Receiver Same Issue Area 9                         & 0.279                            & 0.221                               & 0.346                               & *                                    \\ \hline
Receiver Same Issue Area 10                        & 0.492                            & 0.413                               & 0.570                               & *                                    \\ \hline
Receiver Same Issue Area 11                        & 0.247                            & 0.056                               & 0.411                               & *                                    \\ \hline
Receiver Same Issue Area 12                        & 0.339                            & 0.246                               & 0.430                               & *                                    \\ \hline
Receiver Same Issue Area 13                        & 1.060                            & 0.845                               & 1.239                               & *                                    \\ \hline
Receiver Same Issue Area 14                        & 0.557                            & 0.255                               & 0.793                               & *                                    \\ \hline
Instar(2) $\times$ Sender Year                     & -0.00025                         & -0.00034                            & -0.00017                            & *                                    \\ \hline
Outstar(2) $\times$ Sender Year                    & -0.00036                         & -0.00056                            & -0.00028                            & *                                    \\ \hline
Mutual $\times$ Sender Year                        & -0.010                           & -0.043                              & 0.029                               &                                      \\ \hline
Triangle $\times$ Sender Year                      & -0.0004                          & -0.0021                             & 0.0016                              &                                      \\ \hline
Martin Quinn Score $\times$ Sender Year            & -0.0025                          & -0.0035                             & -0.0013                             & *                                    \\ \hline
Same Issue Area $\times$ Sender Year               & -0.0058                          & -0.0076                             & -0.0043                             & *                                    \\ \hline
Year Difference $\times$ Sender Year               & 0.0004                           & 0.0002                              & 0.0007                              & *                                    \\ \hline
$\text{Year Difference}^2 \times$ Sender Year      & -0.000037                        & -0.000044                           & -0.000030                           & *                                    \\ \hline
MQ Score in Majority $\times$ Sender Year & 0.0013                           & 0.0009                              & 0.0017                              & *                                    \\ \hline
Justices in Majority $\times$ Sender Year   & 0.0029                           & 0.0023                              & 0.0035                              & *                                    \\ \hline
\end{tabular}
\caption{Bootstrapped MPLE Results for the time period $1937-2001$. A '*' indicates that the 2.5th and 97.5th quantile of the variable does not include '0' and as a result is statistically significant.}
\label{my-label}
\end{table}
\newpage


\bibliography{bib} 
\bibliographystyle{apsr}


\end{document}
